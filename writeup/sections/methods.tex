\section*{Methods}
For our baseline model, we had a simple design of a single hidden layer LSTM. Our network takes in a sequence of music notes in ABC notation one at a time. After going through the hidden layer, the output is a softmax of the probability distribution stream of the upcoming musical character. 

\subsection*{Training network using Teacher forcing}
To train our model, we went with a teacher-forcing technique. After passing in a character to the LSTM, we look at the predicted character and correct the weights if necessary. To update our weights, we used categorical cross-entropy since it best fits the scenario. We went with an Adam optimizer to create a model that does a better job at generalizing outputs based on inputs.

\subsection*{Song Generation}


\subsection*{Hyper-parameter Tuning}
Describe the RNN architecture you used in 5.a and briefly describe the approach you took in tuning your hyperparameters.

\subsection*{Feature Evaluation}
Describe why it is important to do feature evaluation. Describe the approach you took in generating the Heatmap of the activation’s of each neurons for each of the characters.